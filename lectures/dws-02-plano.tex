% Created 2024-05-15 qua 12:00
% Intended LaTeX compiler: lualatex
\documentclass{scrartcl}
\usepackage[cm]{fullpage}
\usepackage{mathtools}
\usepackage{fvextra}
\usepackage{fancyvrb}
\usepackage{amsmath}
\usepackage[dvipsnames,table]{xcolor}
\usepackage{listings}
\usepackage{fixltx2e}
\usepackage{fontspec}
\defaultfontfeatures{Ligatures=TeX,Scale=MatchLowercase}
\setmainfont[SmallCapsFont={Linux Libertine Capitals}]{Linux Libertine O}
\setsansfont[SmallCapsFont={Linux Biolinum Capitals}]{Linux Biolinum O}
\setmonofont{Source Code Pro}
\DefineVerbatimEnvironment{verbatim}{Verbatim}{fontsize=\small,formatcom={\color[rgb]{0.5,0,0}},breaklines=true}
\input{/home/jefferson/Sync/tex/custom-listings}

\usepackage{graphicx}
\usepackage{longtable}
\usepackage{wrapfig}
\usepackage{rotating}
\usepackage[normalem]{ulem}
\usepackage{amsmath}
\usepackage{amssymb}
\usepackage{capt-of}
\usepackage{hyperref}
\usepackage{fvextra}
\fvset{breakbefore=(}
\usepackage{syntax}
\setsansfont{DejaVu Sans}
\setmonofont[Scale=MatchLowercase]{DejaVu Sans Mono}
\author{Prof. Jefferson O. Andrade}
\date{\textit{[2024-05-15 qua]}}
\title{Desenvolvimento Web Seguro -- Plano de Aula\\\medskip
\large Semana 2 -- Autenticação e Autorização}
\hypersetup{
 pdfauthor={Prof. Jefferson O. Andrade},
 pdftitle={Desenvolvimento Web Seguro -- Plano de Aula},
 pdfkeywords={},
 pdfsubject={},
 pdfcreator={Emacs 29.2 (Org mode 9.6.15)}, 
 pdflang={English}}
\begin{document}

\maketitle

\section{Objetivos da Aula}
\label{sec:org5dab1b6}
\begin{itemize}
\item Compreender os diferentes métodos de autenticação e autorização.
\item Identificar e resolver problemas comuns relacionados à autenticação e
autorização.
\item Analisar estudos de caso para entender a aplicação prática de métodos seguros
de autenticação e autorização.
\end{itemize}

\section{Estrutura da Aula}
\label{sec:org68e3f16}

\begin{enumerate}
\item \textbf{\textbf{Introdução (5 minutos)}}
\begin{itemize}
\item Breve introdução sobre a importância da autenticação e autorização em
segurança de aplicações web.
\end{itemize}

\item \textbf{\textbf{Apresentações dos Grupos (40-60 minutos)}}
\begin{itemize}
\item \textbf{\textbf{A02:2021 - Cryptographic Failures (10-15 minutos)}}
\begin{itemize}
\item Apresentação por Murilo e Rafael.
\end{itemize}
\item \textbf{\textbf{A03:2021 - Injection (10-15 minutos)}}
\begin{itemize}
\item Apresentação por Marcio e Ana.
\end{itemize}
\item \textbf{\textbf{A07:2021 - Identification and Authentication Failures (10-15 minutos)}}
\begin{itemize}
\item Apresentação por Anderson e Hericca.
\end{itemize}
\item \textbf{\textbf{A10:2021 - Server-Side Request Forgery (10-15 minutos)}}
\begin{itemize}
\item Apresentação por Felipe, Fabrício e Carlos.
\end{itemize}
\end{itemize}

\item \textbf{\textbf{Discussão sobre Apresentações (10 minutos)}}
\begin{itemize}
\item Perguntas e respostas.
\item Feedback sobre as apresentações.
\end{itemize}

\item \textbf{\textbf{Autenticação e Autorização: Conceitos Básicos (20 minutos)}}
\begin{itemize}
\item Definições e diferenças entre autenticação e autorização.
\item Métodos comuns de autenticação (senhas, tokens, biometria, MFA).
\item Mecanismos de autorização
\begin{itemize}
\item Controle de acesso baseado em funções (RBAC -- \emph{Rule Based Access
Control})
\item Controle de acesso baseado em atributos (ABAC -- \emph{Attribute Based Access
Control})
\end{itemize}
\end{itemize}

\item \textbf{\textbf{Estudos de Caso (20 minutos)}}
\begin{itemize}
\item Análise de estudos de caso reais onde falhas na autenticação e autorização
levaram a violações de segurança.
\item Discussão sobre como esses problemas poderiam ter sido evitados.
\end{itemize}

\item \textbf{\textbf{Atividade Prática (30 minutos)}}
\begin{itemize}
\item Resolução de problemas comuns relacionados à autenticação e autorização.
\item Implementação de um sistema simples de autenticação e autorização em um
ambiente de desenvolvimento (exemplo prático).
\end{itemize}

\item \textbf{\textbf{Conclusão e Próximos Passos (10 minutos)}}
\begin{itemize}
\item Recapitulação dos pontos chave da aula.
\item Introdução ao conteúdo da próxima aula.
\item Tarefas pós-aula: Leitura recomendada e exercícios para praticar.
\end{itemize}
\end{enumerate}


\section{Recursos Necessários:}
\label{sec:orga485983}
\begin{itemize}
\item Slides de apresentação.
\item Exemplos de código para a atividade prática.
\item Links para vídeos e leituras recomendadas.
\item Acesso à plataforma de videoconferência (Google Meet) para ministrar a aula
síncrona.
\end{itemize}
\end{document}
