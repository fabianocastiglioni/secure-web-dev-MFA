% Created 2024-05-15 qua 18:34
% Intended LaTeX compiler: lualatex
\documentclass{scrartcl}
\usepackage[cm]{fullpage}
\usepackage{mathtools}
\usepackage{fvextra}
\usepackage{fancyvrb}
\usepackage{amsmath}
\usepackage[dvipsnames,table]{xcolor}
\usepackage{listings}
\usepackage{fixltx2e}
\usepackage{fontspec}
\defaultfontfeatures{Ligatures=TeX,Scale=MatchLowercase}
\setmainfont[SmallCapsFont={Linux Libertine Capitals}]{Linux Libertine O}
\setsansfont[SmallCapsFont={Linux Biolinum Capitals}]{Linux Biolinum O}
\setmonofont{Source Code Pro}
\DefineVerbatimEnvironment{verbatim}{Verbatim}{fontsize=\small,formatcom={\color[rgb]{0.5,0,0}},breaklines=true}
\input{/home/jefferson/Sync/tex/custom-listings}

\usepackage{graphicx}
\usepackage{longtable}
\usepackage{wrapfig}
\usepackage{rotating}
\usepackage[normalem]{ulem}
\usepackage{amsmath}
\usepackage{amssymb}
\usepackage{capt-of}
\usepackage{hyperref}
\usepackage{fvextra}
\fvset{breakbefore=(}
\usepackage{syntax}
\setsansfont{DejaVu Sans}
\setmonofont[Scale=MatchLowercase]{DejaVu Sans Mono}
\author{Prof. Jefferson O. Andrade}
\date{2024-05-15}
\title{Desenvolvimento Web Seguro\\\medskip
\large Semana 02 -- Atividade Prática}
\hypersetup{
 pdfauthor={Prof. Jefferson O. Andrade},
 pdftitle={Desenvolvimento Web Seguro},
 pdfkeywords={},
 pdfsubject={},
 pdfcreator={Emacs 29.2 (Org mode 9.6.15)}, 
 pdflang={English}}
\begin{document}

\maketitle

\section{Parte 1 -- Resolução de Problemas}
\label{sec:orgffb1bf7}

\subsection{Instruções}
\label{sec:org3332097}

Seu grupo deve analisar a aplicação de exemplo, identificar possíveis
vulnerabilidades e discutir possíveis soluções.

\subsubsection{Identificação de Vulnerabilidades:}
\label{sec:org9024aab}
\begin{itemize}
\item Analisar onde e como as senhas são armazenadas no banco de dados.
\item Verificar as rotas e funções que controlam o acesso às funcionalidades da
aplicação.
\end{itemize}

\subsubsection{Correção das Vulnerabilidades:}
\label{sec:org5c3595d}
\begin{itemize}
\item Implementar hashing seguro de senhas usando werkzeug.security (ou outra
biblioteca de hashing).
\item Melhorar o controle de acesso, garantindo que apenas usuários autorizados
possam acessar determinadas funcionalidades.
\end{itemize}


\subsection{Configuração do Ambiente}
\label{sec:org813c669}

\subsubsection{Instalar Python e pip}
\label{sec:org1cd979d}
\begin{itemize}
\item Certifique-se de que você tenha o Python e o pip instalados no seu sistema.
Você pode baixar o Python \href{https://www.python.org/downloads/}{aqui}.
\end{itemize}

\subsubsection{Clone o repositório do Codeberg}
\label{sec:org120b940}
\begin{enumerate}
\item Acesse o \href{https://codeberg.org/profjeffandrade/secure-web-dev}{repositório da disciplina} no \texttt{Codeberg.org}.
\item Clone o repositório em algum local adequado no seu computador local:
\begin{lstlisting}[language=shell,label= ,caption= ,captionpos=b,numbers=none]
git clone https://codeberg.org/profjeffandrade/secure-web-dev.git
\end{lstlisting}
\item Vamos chamar o diretório onde o você clonou o repositório de \texttt{\$SWD}.
\end{enumerate}

\subsubsection{Criar e Ativar um Ambiente Virtual}
\label{sec:orga5263e8}
\begin{enumerate}
\item Navegue até o diretório \texttt{\$SWD/tasks}.
\item Execute os comandos abaixo:
\begin{lstlisting}[language=shell,label= ,caption= ,captionpos=b,numbers=none]
python -m venv venv
source venv/bin/activate  # No Windows, use `venv\Scripts\activate`
\end{lstlisting}
\end{enumerate}

\subsubsection{Instalar Flask e Werkzeug}
\label{sec:org7305874}
\begin{itemize}
\item Execute o comando abaixo:
\begin{lstlisting}[language=shell,label= ,caption= ,captionpos=b,numbers=none]
pip install Flask Werkzeug
\end{lstlisting}
\end{itemize}

\subsubsection{Inicializar o banco de dados.}
\label{sec:orgac167dc}
\begin{itemize}
\item O banco de dados de exemplo (\texttt{example.db}) é iniciado com o usuário \texttt{admin}
(senha \texttt{admin123}). Se precisar apagar o banco de dados e reiniciá-lo, execute
o comando abaixo uma vez:
\begin{lstlisting}[language=shell,label= ,caption= ,captionpos=b,numbers=none]
python app.py
\end{lstlisting}
\end{itemize}

\subsubsection{Rodar a Aplicação:}
\label{sec:org9b8bd6e}
\begin{itemize}
\item Após inicializar o banco de dados, você pode rodar a aplicação com:
\begin{lstlisting}[language=shell,label= ,caption= ,captionpos=b,numbers=none]
flask run
\end{lstlisting}
\end{itemize}

\subsubsection{Acessar a Aplicação}
\label{sec:org95ee108}
\begin{itemize}
\item Abra um navegador e vá para \url{http://127.0.0.1:5000/login} para acessar a
aplicação.
\end{itemize}


\section{Parte 2 -- Implementação}
\label{sec:org16fc0f0}

Para a parte 2, seu grupo deverá desenvolver uma aplicação web simples seguindo
as indicações abaixo:

\begin{enumerate}
\item \textbf{\textbf{Configuração do Ambiente:}}
\begin{itemize}
\item Configure um ambiente de desenvolvimento usando uma linguagem de
programação e framework de sua escolha (por exemplo, Python com Flask, ou
JavaScript com Node.js).
\end{itemize}
\item \textbf{\textbf{Autenticação:}}
\begin{itemize}
\item Implemente um sistema de login básico utilizando nome de usuário e senha.
\item Adicione a funcionalidade de autenticação multifator (MFA) para maior segurança.
\end{itemize}
\item \textbf{\textbf{Autorização:}}
\begin{itemize}
\item Crie diferentes níveis de permissão (por exemplo, usuário comum,
administrador).
\item Implemente controles de acesso para restringir ações baseadas nas
permissões do usuário.
\end{itemize}
\end{enumerate}
\end{document}
