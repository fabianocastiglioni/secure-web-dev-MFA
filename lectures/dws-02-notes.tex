% Created 2024-05-15 qua 14:11
% Intended LaTeX compiler: lualatex
\documentclass{scrartcl}
\usepackage[margin=2cm,includefoot,footskip=30pt]{geometry}\usepackage{layout}\usepackage{mathtools}
\usepackage{fvextra}
\usepackage{fancyvrb}
\usepackage{amsmath}
\usepackage[dvipsnames,table]{xcolor}
\usepackage{listings}
\usepackage{fixltx2e}
\usepackage{fontspec}
\defaultfontfeatures{Ligatures=TeX,Scale=MatchLowercase}
\setmainfont[SmallCapsFont={Linux Libertine Capitals}]{Linux Libertine O}
\setsansfont[SmallCapsFont={Linux Biolinum Capitals}]{Linux Biolinum O}
\setmonofont{Source Code Pro}
\DefineVerbatimEnvironment{verbatim}{Verbatim}{fontsize=\small,formatcom={\color[rgb]{0.5,0,0}},breaklines=true}
\input{/home/jefferson/Sync/tex/custom-listings}

\usepackage{graphicx}
\usepackage{longtable}
\usepackage{wrapfig}
\usepackage{rotating}
\usepackage[normalem]{ulem}
\usepackage{amsmath}
\usepackage{amssymb}
\usepackage{capt-of}
\usepackage{hyperref}
\usepackage{fvextra}
\fvset{breakbefore=(}
\usepackage{syntax}
\setsansfont{DejaVu Sans}
\setmonofont[Scale=MatchLowercase]{DejaVu Sans Mono}
\author{Prof. Jefferson O. Andrade}
\date{2024-05-15}
\title{Desenvolvimento Web Seguro -- Notas de Aula\\\medskip
\large Semana 02 -- Autenticação e Autorização}
\hypersetup{
 pdfauthor={Prof. Jefferson O. Andrade},
 pdftitle={Desenvolvimento Web Seguro -- Notas de Aula},
 pdfkeywords={},
 pdfsubject={},
 pdfcreator={Emacs 29.2 (Org mode 9.6.15)}, 
 pdflang={English}}
\begin{document}

\maketitle
\setcounter{tocdepth}{2}
\tableofcontents


\section{Introdução}
\label{sec:orgf5f04a1}

A autenticação e a autorização são componentes críticos na segurança de
aplicações web. Eles garantem que somente usuários legítimos possam acessar o
sistema e que esses usuários só possam realizar ações para as quais estão
autorizados. Aqui estão algumas razões pelas quais a autenticação e a
autorização são essenciais:

\begin{enumerate}
\item \textbf{\textbf{Proteção de Dados Sensíveis:}}
\begin{itemize}
\item A autenticação garante que apenas usuários autorizados possam acessar
informações sensíveis, como dados pessoais, financeiros e comerciais. Sem
autenticação adequada, esses dados podem ser facilmente acessados por
indivíduos não autorizados, levando a vazamentos de informações e
comprometimento de privacidade.
\end{itemize}

\item \textbf{\textbf{Prevenção de Acesso Não Autorizado:}}
\begin{itemize}
\item A autorização assegura que os usuários, mesmo que autenticados, só possam
realizar ações permitidas com base em suas permissões. Isso impede que
usuários com permissões limitadas acessem ou modifiquem recursos críticos
do sistema, como configurações administrativas ou dados de outros usuários.
\end{itemize}

\item \textbf{\textbf{Redução de Riscos de Ataques:}}
\begin{itemize}
\item Implementações fracas de autenticação e autorização são alvos comuns de
ataques cibernéticos, como força bruta, roubo de credenciais e
escalonamento de privilégios. Métodos robustos de autenticação, como a
autenticação multifator (MFA), e modelos de autorização bem definidos, como
RBAC (Role-Based Access Control), ajudam a mitigar esses riscos.
\end{itemize}

\item \textbf{\textbf{Conformidade com Regulamentações:}}
\begin{itemize}
\item Muitas regulamentações de proteção de dados e privacidade, como o GDPR
(Regulamento Geral sobre a Proteção de Dados) e a LGPD (Lei Geral de
Proteção de Dados), exigem que as organizações implementem medidas
adequadas de autenticação e autorização para proteger os dados dos
usuários. A conformidade com essas regulamentações é crucial para evitar
penalidades legais e manter a confiança dos clientes.
\end{itemize}

\item \textbf{\textbf{Manutenção da Integridade do Sistema:}}
\begin{itemize}
\item A autenticação e a autorização protegem a integridade do sistema ao
garantir que apenas usuários legítimos possam realizar ações que afetam o
funcionamento e a segurança da aplicação. Isso inclui prevenir que
atacantes introduzam códigos maliciosos ou façam alterações não autorizadas
no sistema.
\end{itemize}
\end{enumerate}

A autenticação e a autorização são pilares fundamentais na construção de
aplicações web seguras. Elas não apenas protegem os dados e a privacidade dos
usuários, mas também garantem a integridade e a confiabilidade do sistema como
um todo.

\section{Autenticação e Autorização}
\label{sec:org5972197}
\subsection{Definição de Autenticação:}
\label{sec:orgd82fc1a}
Autenticação é o processo de verificar a identidade de um usuário ou sistema. Em
outras palavras, é a etapa onde o sistema confirma que o usuário é quem ele
afirma ser. Isso geralmente é feito por meio de credenciais, como nome de
usuário e senha, mas também pode envolver métodos mais avançados, como biometria
(impressões digitais, reconhecimento facial) ou autenticação multifator (MFA),
que combina vários métodos de verificação.

\subsection{Definição de Autorização:}
\label{sec:org5ca1143}
Autorização é o processo de conceder a um usuário ou sistema permissão para
acessar recursos específicos ou realizar certas ações dentro de um sistema.
Enquanto a autenticação verifica a identidade, a autorização determina o que
essa identidade pode fazer. Por exemplo, após um usuário ser autenticado, o
sistema verifica suas permissões para acessar arquivos, executar comandos ou
modificar dados.

\subsection{Diferenças Entre Autenticação e Autorização:}
\label{sec:orgf54cfc7}

\begin{enumerate}
\item \textbf{\textbf{Propósito:}}
\begin{itemize}
\item \textbf{\textbf{Autenticação:}} Verifica a identidade do usuário. É o processo de garantir que o usuário é quem ele diz ser.
\item \textbf{\textbf{Autorização:}} Controla o acesso aos recursos e ações dentro do sistema. Determina o que o usuário autenticado pode fazer.
\end{itemize}

\item \textbf{\textbf{Etapas do Processo:}}
\begin{itemize}
\item \textbf{\textbf{Autenticação:}} Ocorre primeiro. Sem autenticação, o sistema não pode determinar quais permissões conceder ao usuário.
\item \textbf{\textbf{Autorização:}} Ocorre após a autenticação. Somente usuários autenticados podem ser autorizados a acessar recursos ou executar ações específicas.
\end{itemize}

\item \textbf{\textbf{Métodos Comuns:}}
\begin{itemize}
\item \textbf{\textbf{Autenticação:}} Nome de usuário e senha, tokens, biometria, autenticação multifator (MFA).
\item \textbf{\textbf{Autorização:}} Controle de Acesso Baseado em Funções (RBAC), Controle de Acesso Baseado em Atributos (ABAC), listas de controle de acesso (ACLs).
\end{itemize}

\item \textbf{\textbf{Papel na Segurança:}}
\begin{itemize}
\item \textbf{\textbf{Autenticação:}} Protege o sistema contra acesso não autorizado verificando identidades.
\item \textbf{\textbf{Autorização:}} Protege recursos específicos e funcionalidades do sistema, garantindo que somente usuários autorizados possam acessá-los ou executá-los.
\end{itemize}
\end{enumerate}

\subsection{Exemplo Prático:}
\label{sec:org489bab7}
\begin{itemize}
\item \textbf{\textbf{Autenticação:}} Quando um usuário faz login em um site com seu nome de
usuário e senha, o sistema verifica essas credenciais para confirmar a
identidade do usuário.
\item \textbf{\textbf{Autorização:}} Após o login, o sistema verifica as permissões do usuário.
Por exemplo, um usuário comum pode acessar seu próprio perfil e dados,
enquanto um administrador pode ter acesso a funções adicionais, como
gerenciamento de usuários e configurações do sistema.
\end{itemize}

A autenticação e a autorização são processos distintos, mas complementares, que
juntos garantem que somente usuários legítimos e autorizados possam acessar
recursos e executar ações dentro de um sistema.

\section{Métodos Comuns de Autenticação}
\label{sec:org4f301dd}
\subsection{Nome de Usuário e Senha}
\label{sec:org6b41dcc}
\begin{itemize}
\item \textbf{\textbf{Descrição:}} Este é o método mais comum de autenticação, onde os usuários
fornecem um nome de usuário (ou ID) e uma senha para acessar um sistema.
\item \textbf{\textbf{Vantagens:}} Simples e fácil de implementar.
\item \textbf{\textbf{Desvantagens:}} Vulnerável a ataques de força bruta, phishing e
reutilização de senhas. Requer políticas fortes de criação e gerenciamento
de senhas para ser eficaz.
\end{itemize}

\subsection{Tokens de Autenticação}
\label{sec:org1924b6b}
\begin{itemize}
\item \textbf{\textbf{Descrição:}} Tokens são gerados pelo servidor e enviados ao cliente após
a autenticação inicial. Os tokens são então usados para autenticar
solicitações subsequentes sem necessidade de enviar novamente o nome de
usuário e senha.
\item \textbf{\textbf{Tipos Comuns:}} JWT (JSON Web Tokens), tokens de sessão.
\item \textbf{\textbf{Vantagens:}} Reduz a exposição das credenciais do usuário e pode ser
usado para autenticação sem estado (stateless authentication).
\item \textbf{\textbf{Desvantagens:}} Necessita de mecanismos seguros para geração,
armazenamento e invalidação de tokens.
\end{itemize}

\subsection{Biometria}
\label{sec:orgd482624}
\begin{itemize}
\item \textbf{\textbf{Descrição:}} Usa características físicas ou comportamentais únicas do
usuário, como impressões digitais, reconhecimento facial ou reconhecimento
de voz, para autenticar a identidade.
\item \textbf{\textbf{Vantagens:}} Difícil de falsificar, elimina a necessidade de memorizar
senhas.
\item \textbf{\textbf{Desvantagens:}} Pode ser invasivo em termos de privacidade, requer
hardware especializado, e pode ser vulnerável a ataques sofisticados.
\end{itemize}

\subsection{Autenticação Multifator (MFA)}
\label{sec:orgba188b1}
\begin{itemize}
\item \textbf{\textbf{Descrição:}} Combina dois ou mais métodos de autenticação, como algo que
o usuário conhece (senha), algo que o usuário possui (token), e algo que o
usuário é (biometria).
\item \textbf{\textbf{Vantagens:}} Significativamente mais seguro que métodos de autenticação
únicos, pois um invasor teria que comprometer múltiplos fatores para obter
acesso.
\item \textbf{\textbf{Desvantagens:}} Pode ser mais complexo e caro de implementar, pode causar
inconveniência ao usuário.
\end{itemize}

\subsection{Autenticação Baseada em Certificados}
\label{sec:orged876ac}
\begin{itemize}
\item \textbf{\textbf{Descrição:}} Usa certificados digitais para autenticar a identidade do
usuário ou dispositivo. Os certificados são emitidos por uma Autoridade
Certificadora (CA) confiável.
\item \textbf{\textbf{Vantagens:}} Alta segurança, pode ser usado para autenticação de
dispositivos além de usuários.
\item \textbf{\textbf{Desvantagens:}} Requer uma infraestrutura de gerenciamento de
certificados (PKI), pode ser complexo de administrar.
\end{itemize}

\subsection{Autenticação OAuth/OpenID Connect}
\label{sec:org25b72c6}
\begin{itemize}
\item \textbf{\textbf{Descrição:}} Permite que os usuários autentiquem-se usando contas de
terceiros (como Google, Facebook) sem precisar criar novas credenciais.
\item \textbf{\textbf{Vantagens:}} Conveniente para usuários, reduz a necessidade de gerenciar
múltiplas senhas.
\item \textbf{\textbf{Desvantagens:}} Depende da segurança e disponibilidade do provedor de
identidade terceiro.
\end{itemize}

\section{Mecanismos de Autorização}
\label{sec:org2bcb5e1}
\subsection{Controle de Acesso Baseado em Funções (RBAC)}
\label{sec:org9c02cb5}
\subsubsection{Definição}
\label{sec:orgbbbcade}
O Controle de Acesso Baseado em Funções (Role-Based Access Control - RBAC) é um
modelo de controle de acesso onde as permissões são atribuídas a funções
específicas dentro de uma organização, e os usuários são então atribuídos a
essas funções. Cada função tem um conjunto definido de permissões que determinam
o que os usuários naquela função podem acessar e realizar dentro do sistema.

\subsubsection{Características Principais}
\label{sec:org1bdd82c}
\begin{itemize}
\item \textbf{\textbf{Funções:}} Representam um conjunto de permissões associadas a uma função
específica dentro da organização (por exemplo, administrador, editor,
visualizador).
\item \textbf{\textbf{Permissões:}} São as ações que podem ser realizadas dentro do sistema (por
exemplo, ler, escrever, excluir).
\item \textbf{\textbf{Usuários:}} São atribuídos a uma ou mais funções, herdando as permissões
associadas a essas funções.
\end{itemize}

\subsubsection{Vantagens}
\label{sec:org226f027}
\begin{itemize}
\item \textbf{\textbf{Facilidade de Gerenciamento:}} Simplifica a administração de permissões,
pois as permissões são atribuídas a funções e não diretamente a usuários.
\item \textbf{\textbf{Escalabilidade:}} Facilita a adição de novos usuários e a atribuição de
permissões à medida que a organização cresce.
\item \textbf{\textbf{Segurança:}} Reduz a possibilidade de erro humano na atribuição de
permissões, garantindo que os usuários tenham apenas as permissões necessárias
para suas funções.
\end{itemize}

\subsubsection{Desvantagens}
\label{sec:org81b5147}
\begin{itemize}
\item \textbf{\textbf{Rigidez:}} Pode ser inflexível em cenários onde os requisitos de acesso são
dinâmicos e mudam frequentemente.
\item \textbf{\textbf{Manutenção:}} Requer manutenção constante para garantir que as funções e
suas permissões estejam atualizadas e alinhadas com as necessidades da
organização.
\end{itemize}

\subsection{Controle de Acesso Baseado em Atributos (ABAC)}
\label{sec:orgd5f10a4}
\subsubsection{Definição}
\label{sec:org584c86c}
O Controle de Acesso Baseado em Atributos (Attribute-Based Access Control -
ABAC) é um modelo de controle de acesso onde as permissões são atribuídas com
base em um conjunto de atributos e políticas que determinam as condições sob as
quais o acesso é permitido. Os atributos podem incluir características do
usuário, do recurso ou do ambiente.

\subsubsection{Características Principais}
\label{sec:orge4a425c}
\begin{itemize}
\item \textbf{\textbf{Atributos:}} Podem incluir atributos do usuário (por exemplo, função,
departamento), atributos do recurso (por exemplo, tipo de documento,
classificação), e atributos do ambiente (por exemplo, horário, localização).
\item \textbf{\textbf{Políticas:}} Definem as regras que determinam as condições sob as quais o
acesso é permitido ou negado. As políticas são escritas em termos de atributos
e suas combinações.
\item \textbf{\textbf{Decisões de Acesso:}} São feitas em tempo real com base na avaliação dos
atributos e nas políticas definidas.
\end{itemize}

\subsubsection{Vantagens}
\label{sec:orgcb5c282}
\begin{itemize}
\item \textbf{\textbf{Flexibilidade:}} Oferece um controle de acesso mais granular e dinâmico,
adaptando-se facilmente a mudanças nos requisitos de acesso.
\item \textbf{\textbf{Segurança:}} Permite a implementação de políticas de segurança complexas que
consideram múltiplos fatores, melhorando a precisão das decisões de acesso.
\item \textbf{\textbf{Personalização:}} Permite personalizar o acesso com base em uma ampla
variedade de atributos, proporcionando um controle mais detalhado.
\end{itemize}

\subsubsection{Desvantagens}
\label{sec:org9e6accd}
\begin{itemize}
\item \textbf{\textbf{Complexidade:}} Pode ser mais complexo de implementar e gerenciar devido à
necessidade de definir e manter políticas e atributos.
\item \textbf{\textbf{Desempenho:}} A avaliação em tempo real de atributos e políticas pode
impactar o desempenho, especialmente em sistemas com grande volume de acessos.
\end{itemize}

\section{Estudos de Caso}
\label{sec:orga681c29}

\subsection{Caso Target (2013)}
\label{sec:org05937ab}
\begin{itemize}
\item \textbf{\textbf{Descrição:}} Em 2013, a Target sofreu uma das maiores violações de dados da
história, resultando no comprometimento de informações de cartão de crédito e
débito de aproximadamente 40 milhões de clientes.
\item \textbf{\textbf{Problema:}} Os atacantes exploraram credenciais de um fornecedor de HVAC
(aquecimento, ventilação e ar condicionado) para acessar a rede da Target. Uma
vez dentro, eles conseguiram se mover lateralmente pela rede devido a falhas
de autorização inadequadas.
\item \textbf{\textbf{Consequências:}} Perda financeira significativa, dano à reputação e custos
legais elevados.
\item \textbf{\textbf{Lições Aprendidas:}} Implementação de autenticação multifator (MFA) e
segmentação de rede para limitar o acesso a partes críticas do sistema.
\end{itemize}

\subsection{Caso Uber (2016)}
\label{sec:org833ffe5}
\begin{itemize}
\item \textbf{\textbf{Descrição:}} Em 2016, a Uber sofreu uma violação de dados que expôs
informações pessoais de 57 milhões de motoristas e passageiros.
\item \textbf{\textbf{Problema:}} Os atacantes conseguiram acessar uma base de dados armazenada na
AWS usando credenciais de acesso que foram acidentalmente expostas em um
repositório de código no GitHub.
\item \textbf{\textbf{Consequências:}} Pagamento de US\$ 148 milhões em um acordo judicial, além de
danos à reputação.
\item \textbf{\textbf{Lições Aprendidas:}} Uso de práticas seguras de gerenciamento de credenciais
e autenticação multifator para acessar recursos sensíveis.
\end{itemize}

\subsection{Caso Equifax (2017)}
\label{sec:org2b61d3d}
\begin{itemize}
\item \textbf{\textbf{Descrição:}} Em 2017, a Equifax sofreu uma violação massiva de dados que
expôs informações pessoais de 147 milhões de pessoas.
\item \textbf{\textbf{Problema:}} A vulnerabilidade inicial foi uma falha de autorização em um
portal web da Equifax. A falta de segmentação adequada e de controles de
autorização permitiu que os atacantes acessassem grandes volumes de dados.
\item \textbf{\textbf{Consequências:}} Multas significativas, perda de confiança do consumidor e
altos custos de mitigação e reparação.
\item \textbf{\textbf{Lições Aprendidas:}} Implementação de controles de autorização robustos e
segmentação de rede para limitar o acesso a dados sensíveis.
\end{itemize}

\subsection{Caso Facebook (2018)}
\label{sec:orga571a87}
\begin{itemize}
\item \textbf{\textbf{Descrição:}} Em 2018, o Facebook revelou uma violação que afetou cerca de 50
milhões de contas de usuários.
\item \textbf{\textbf{Problema:}} Uma falha no sistema de autenticação permitiu que atacantes
roubassem tokens de acesso, que poderiam ser usados para assumir o controle
das contas dos usuários.
\item \textbf{\textbf{Consequências:}} Exposição de informações pessoais, investigações
regulatórias e danos à reputação.
\item \textbf{\textbf{Lições Aprendidas:}} Revisão e fortalecimento dos mecanismos de autenticação
e implementação de monitoramento contínuo para detectar atividades suspeitas.
\end{itemize}

\subsection{Caso Twitter (2020)}
\label{sec:orgf01d17e}
\begin{itemize}
\item \textbf{\textbf{Descrição:}} Em 2020, várias contas de alto perfil no Twitter foram
comprometidas em um ataque de engenharia social.
\item \textbf{\textbf{Problema:}} Os atacantes enganaram os funcionários do Twitter para que
revelassem suas credenciais, permitindo acesso às ferramentas internas de
administração.
\item \textbf{\textbf{Consequências:}} Controle temporário de contas de alto perfil, divulgação de
mensagens fraudulentas e investigação por agências governamentais.
\item \textbf{\textbf{Lições Aprendidas:}} Educação contínua dos funcionários sobre engenharia
social e implementação de autenticação multifator para acessos
administrativos.
\end{itemize}
\end{document}
